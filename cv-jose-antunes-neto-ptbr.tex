% !TeX program = xelatex
%& -job-name=cvportugues

\documentclass[a4paper,10pt]{article}

\usepackage{preamble/packages}
\usepackage{preamble/theme}

\begin{document}

% Título ----------------------------------------------------------------------
\par{
    \centering
    {\Huge \textsc{Antunes Neto}, José
    }\bigskip\par
}

% Dados pessoais --------------------------------------------------------------
\section{Dados Pessoais}

\begin{tabular}{rp{10.75cm}}
    \textsc{Nascimento:} & Itabirito, Brasil | 3 de abril de 1996 \\
    \textsc{Endereço (EUA):}   & 2211 Campus Drive, Office 4348. Evanston, IL. 60208 \\
    \textsc{Endereço (BR):}& Rua Belo Vale, 125, Matozinhos. Itabirito-MG, Brasil. 35452-068 \\
    \textsc{Telefone:}     & \href{tel:+17733121395}{+1 (773) 312-1395} (EUA) \quad \href{tel:+5531989301667}{+55 (31) 98930-1667} (BR) \\
    \textsc{E-mail:}     & \href{mailto:jose.neto@kellogg.northwestern.edu}{jose.neto@kellogg.northwestern.edu} \\
    \textsc{URL:}   & \href{http://joseparreiras.github.io}{joseparreiras.github.io} \\
    \textsc{LinkedIn:}  & \href{https://www.linkedin.com/in/jose-antunes-neto/}{in/jose-antunes-neto}
\end{tabular}

% Bio -------------------------------------------------------------------------
\section{Resumo Profissional}

\noindent
Atualmente, estou em busca de oportunidades nas áreas de pesquisa quantitativa e análise de risco em bancos e fundos de investimento. Sou doutorando em Finanças na Kellogg School of Management – Northwestern University, com sólida experiência em pesquisa empírica e forte formação quantitativa. Minha trajetória acadêmica e profissional é centrada em precificação de ativos, econometria financeira e macrofinanças, unindo modelagem econômica rigorosa à análise de dados.

Colaborei com instituições financeiras e centros de pesquisa no desenvolvimento de bases de dados de alta frequência, construção de modelos preditivos e inferência estatística em ambientes incertos. Tenho domínio em Python e R, com experiência adicional em SQL, Git e \LaTeX{}, e amplo conhecimento em métodos quantitativos e teoria financeira. Também publiquei pacotes em Python para modelagem estatística e visualização de dados, disponíveis no meu GitHub (\href{https://github.com/joseparreiras}{github.com/joseparreiras}). Meu objetivo é aplicar esse conjunto de habilidades em funções voltadas à pesquisa no setor financeiro, contribuindo para estratégias de investimento, modelagem de risco e análise de dados.

% Formação Acadêmica ----------------------------------------------------------
\begin{samepage}

\section{Educação}
\begin{tabular}{r|p{10.75cm}}
    \textsc{Ago} 2020 - \parbox[t]{1.5cm}{\centering \textsc{Jun} 2026\\ (Previsto)} & Ph.D. em Finanças, \normalsize\textbf{Kellogg School of Management – Northwestern University} \\
    & \textsc{Finanças} \textit{(Econometria e Economia Quantitativa)} \\
    & \normalsize Programa com designação \textsc{STEM} \\
    & \normalsize \textsc{Média}: 3.52 de 4 \\
    \textsc{Jan} 2018 - \textsc{Mai} 2020 & Mestrado em Economia, \normalsize\textbf{Escola de Economia de São Paulo – FGV} \\
    & \textsc{Economia} \\
    & \normalsize \textsc{Dissertação}: \href{http://hdl.handle.net/10438/29188}{``Testing for Long-Memory Common Features in Volatility Processes"} | \small Orientador: Prof. Marcelo Fernandes \\
    & \normalsize \textsc{Média}: 8.53 de 10 \\
    \textsc{Jan} 2014 - \textsc{Dez} 2017 & Bacharelado em Economia, \normalsize\textbf{IBMEC-MG}, Belo Horizonte \\
    & \normalsize \textsc{Monografia}: “Análise da Evolução da Dívida Europeia” | \small Orientador: Prof. Arilton Teixeira \\
    & Entre os 3 melhores alunos da turma \\
    & \textsc{Média}: 90.60 de 100
\end{tabular}

% Experiência Profissional -----------------------------------------------------
\section{Experiência Profissional}
\begin{tabular}{r|p{10.75cm}}
    \textsc{Out 2024 - Mar 2025} & Consultor de Pesquisa na Jubarte Capital \\
    & \footnotesize{Coleta de dados e construção de base própria de mercados futuros de renda fixa no Brasil, com dados da B3 e ANBIMA. Desenvolvimento de ativos sintéticos para apoio a projetos de pesquisa.} \\
    \textsc{Ago 2021 - Jun 2026} & Ph.D. Student Researcher na Kellogg School of Management \\
    & \footnotesize{Pesquisa independente nas áreas de precificação empírica de ativos e econometria financeira.} \\
    \textsc{Mar 2020 - Mar 2022} & Pesquisador no \href{https://covid19analytics.com.br/}{COVID-19 Analytics} \\
    & \footnotesize{Coleta e análise de dados da pandemia no Brasil. Estimativa de modelos para projeção de casos e monitoramento do número de reprodução (R0).} \\
    \textsc{Jan 2020 - Jul 2020} & Assistente de Pesquisa na B3 – Bolsa de Valores do Brasil \\
    & \footnotesize{Apoio à pesquisa com os professores Marcelo Fernandes, Bruno Giovanetti e Fernando Chague (FGV/CEQEF).} \\
    & \footnotesize{Análise do mercado acionário brasileiro, com foco no comportamento e modelagem da atuação de traders de alta frequência.} \\
    \textsc{Out 2018 - Fev 2019} & Assistente de Pesquisa na FEBRABAN \\
    & \footnotesize{Apoio ao projeto ``Investigating the Dynamics of Lending and Money Market Interest Rates in Brazil: A closer look to disaggregated data”, com os professores Pedro Valls Pereira (FGV/CEQEF) e Emerson Marçal (FGV/CEMAP).} \\
    & \footnotesize{Construção de base de dados, estimação de modelos e programação.}
\end{tabular}
\end{samepage}

% Pesquisa ---------------------------------------------------------------------
\clearpage
\begin{samepage}
\section{Pesquisa}
\subsection{Publicações}
\begin{itemize}[label={}]
    \item Testing for Long-Memory Common Features in Volatility Processes \\
        \textbf{com Marcelo Fernandes} \\
        \textit{FGV EESP – CME: Dissertações, Mestrado em Economia, 2020}
\end{itemize}
\subsection{Trabalhos em Andamento}
\begin{itemize}
    \item The Impact of Information Shocks in the Dispersion of Betas. \href{https://joseparreiras.github.io/projects/news-and-betas}{Link}
    \item Duration of Stock Market Crashes. \href{https://joseparreiras.github.io/projects/news-and-betas}{Link} \\
        \textbf{com Ravi Jagannathan}
    \item Volatility Timing with Option Based Measures. \href{https://joseparreiras.github.io/projects/news-and-betas}{Link}
\end{itemize}

% % Ensino -----------------------------------------------------------------------
% \section{Experiência Docente}
% \subsection{Kellogg School of Management}
% \begin{tabular}{r|p{10.5cm}}
%     \textsc{Fall} 2021 & FINC-450: Capital Markets \\
%     & Monitor da disciplina com o Prof. Erez Levy \\
%     \textsc{Fall} 2021 & KELLG\_FE-312: Investments \\
%     & Monitor da disciplina com o Prof. Erez Levy \\
%     \textsc{Spring} 2023 & FINC-585: Asset Pricing III \\
%     & Monitor da disciplina com o Prof. Torben Andersen
% \end{tabular}

% Bolsas -----------------------------------------------------------------------
\section{Bolsas e Premiações}
\begin{tabular}{r|p{10.5cm}}
    \textsc{Jan} 2018 - \textsc{Fev} 2018 & Bolsa de Pesquisa da EESP-FGV \\
    \textsc{Mar} 2018 - \textsc{Mar} 2020 & Bolsa de Pesquisa da CAPES
\end{tabular}

% Idiomas ----------------------------------------------------------------------
\section{Idiomas}
\begin{tabular}{r|l}
    Português & Nativo \\
    Inglês & Fluente \\
    Espanhol & Conhecimento intermediário
\end{tabular}

% Habilidades Computacionais ---------------------------------------------------
\section{Skills}
\begin{tabular}{r|l}
    Avançado & \textsc{Python, R} \\
    Intermediário & \LaTeX{}, \textsc{SQL, Git, Web Scraping, Microsoft Office} \\
    Básico & \textsc{Stata, Cloud Computing}
\end{tabular}
\end{samepage}

\end{document}